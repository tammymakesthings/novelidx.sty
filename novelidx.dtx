% \iffalse meta-comment
%
% Copyright (C) 2021 by Tammy Cravit <tammymakesthings@gmail.com>
% ---------------------------------------------------------------------------
% This work may be distributed and/or modified under the
% conditions of the LaTeX Project Public License, either version 1.3
% of this license or (at your option) any later version.
% The latest version of this license is in
%   http://www.latex-project.org/lppl.txt
% and version 1.3 or later is part of all distributions of LaTeX
% version 2005/12/01 or later.
%
% This work has the LPPL maintenance status `maintained'.
%
% The Current Maintainer of this work is <Tammy Cravit>.
%
% This work consists of the files novelidx.dtx and novelidx.ins
% and the derived filebase novelidx.sty.
%
% \fi
%
% \iffalse
%<*driver>
\ProvidesFile{novelidx.dtx}
%</driver>
%<package>\NeedsTeXFormat{LaTeX2e}[1999/12/01]
%<package>\ProvidesPackage{novelidx}
%<*package>
    [2021-10-17 0.1.0 Indexing helpers for novel writing]
%</package>
%
%<*driver>
\documentclass[letterpaper]{ltxdoc}
\usepackage{tabularx}
\usepackage{hyperref}
\usepackage{novelidx}[2021/10/17]
\EnableCrossrefs
\CodelineIndex
\RecordChanges
\begin{document}
  \DocInput{novelidx.dtx}
  \PrintChanges
  \PrintIndex
\end{document}
%</driver>
% \fi
%
% \CheckSum{555}
%
% \CharacterTable
%  {Upper-case    \A\B\C\D\E\F\G\H\I\J\K\L\M\N\O\P\Q\R\S\T\U\V\W\X\Y\Z
%   Lower-case    \a\b\c\d\e\f\g\h\i\j\k\l\m\n\o\p\q\r\s\t\u\v\w\x\y\z
%   Digits        \0\1\2\3\4\5\6\7\8\9
%   Exclamation   \!     Double quote  \"     Hash (number) \#
%   Dollar        \$     Percent       \%     Ampersand     \&
%   Acute accent  \'     Left paren    \(     Right paren   \)
%   Asterisk      \*     Plus          \+     Comma         \,
%   Minus         \-     Point         \.     Solidus       \/
%   Colon         \:     Semicolon     \;     Less than     \<
%   Equals        \=     Greater than  \>     Question mark \?
%   Commercial at \@     Left bracket  \[     Backslash     \\
%   Right bracket \]     Circumflex    \^     Underscore    \_
%   Grave accent  \`     Left brace    \{     Vertical bar  \|
%   Right brace   \}     Tilde         \~}
%
%
% \changes{0.1.0}{2021-10-17}{Converted to DTX file}
%
% \DoNotIndex{\newcommand,\newenvironment}
%
% \providecommand*{\url}{\texttt}
% \GetFileInfo{novelidx.dtx}
% \title{The \textsf{novelidx} package}
% \author{Tammy Cravit \\ \url{tammymakesthings@gmail.com}}
% \date{\fileversion~from \filedate}
%
% \maketitle
%
% \section{Introduction}
%
% \LaTeX is suitable for publishing all sorts of documents --- including novels.
% Using the appropriate tools, \LaTeX can produce printed books, but also e-books
% in a variety of formats.
%
% The \LaTeX toolset can help authors with other tasks beyond just creating
% finished documents, though. One of the challenges I face as a writer of series
% novels is keeping track of details. ``Where did I say my main character was
% born?'' ``What color was the getaway car?'' Questions like these can vex
% writers fiction. The \textsf{novelidx} package was intended to help solve this
% problem.
%
% Using the indexing facilities of \LaTeX I can mark such pieces of information
% in my manuscript. I can then generate indexes of important characters, settings,
% plot points, and so forth. These allow me to print a ``series bible'' version
% of my books that can help me keep these details straight.
%
% One could easily accomplish this task using just the \textsf{imakeidx} \LaTeX
% package. In fact, this is how I initially solved this problem. But then I
% discovered that I had many inconsistently named and formatted indexing entries
% in my manuscript. And too, keeping these index entries consistent when you
% have to add multiple index entries to your manuscript was a constant challenge.
%
% The \textsf{novelidx} package attempts to simplify this by providing a set of
% macros that, in turn, define commands which can be used to insert the
% appropriate index entry markers at each relevant point in the document.
%
% \section{Setting Up Your Document}
%
% The basic setup of the document is simple: Just add a
% \verb|\usepackage{novelidx}| statement to your document to import the
% package. Then, use the macros in the package to define your indexing
% commands. Lastly, use the indexing commands you've defined (and others, if
% necessary) to notate the important characters, locations, and so forth in
% your manuscript.
%
% \DescribeMacro{\SetupNovelIndexes}
%
% The \verb|\SetupNovelIndexes| macro sets up the indexes for your manuscript.
% We use the \verb|AtEndPreamble| macro from \texttt{etoolbox} to automatically
% invoke the macro, so you shouldn't have to call it directly. It creates the
% following indexes in your document:
%
% \begin{description}
%     \item[\texttt{char}] Indexes characters and their relationships, as well as
%     character-focused plot points and locations.
%     \item[\texttt{setting}] Indexes locations throughout the manuscript.
%     \item[\texttt{plot}] Indexes plot points and plot information.
%     \item[\texttt{revision}] Indexes revision notes in the manuscript.
%     \item[\texttt{research}] Indexes research notes in the manuscript.
%     \item[\texttt{master}] Holds the ``master'' index, a cross-reference of all
%     the information about your manuscript.
% \end{description}
%
% Although the \textsf{novelidx} package provides macros to help you build
% indexing commands, it leverages the capabilities of the \textsf{lmakeidx}
% package to do so. This means you can alos manually add entries to your index
% using the \verb|\index| command provided by that package, adding your entries
% to the appropriate index(es) as necessary.
%
% \section{Defining \& Using the Indexing Macros}
%
% The \textsf{novelidx} package provides macros which create indexing commands
% for the characters, settings, plot points, and so forth in your work. This
% simplifies the process of adding index markings to your manuscript. There are
% a variety of macros for defining indexing commands, and the index marking
% commands they create follow a similar nomenclature.
%
% Each command definition macro in the \textsf{novelidx} package creates two
% commands. The ``standard'' form of the command adds appropriate entries to
% the relevant indexes, and takes no arguments. An ``extra'' form of each
% command, with an ``X'' appended to its name, allows you to pass optional
% extra information into the index. How this extra information is rendered
% depends on the specific command being used.
%
% For clarity, we'll refer to the commands contained in the \textsf{novelidx}
% package --- those that \emph{define} the indexing terms --- as ``index
% creation macros'', and the commands which these macros define as ``index
% marking macros''.
%
% The index marking commands created by the package follow a consistent 
% naming pattern, as follows:
%
% \begin{center}
% \begin{tabularx}{\textwidth}{|l|l|ll|}
% \hline
% \textbf{Entry Type} & \textbf{Subtype}             & \textbf{``Standard'' Prefix} & \textbf{``Extra'' Prefix} \\
% \hline
% \textbf{Character}  & Normal Character             & \texttt{CHAR}                & \texttt{CHARX}            \\
% \textbf{Character}  & Romantic Relationship        & \texttt{ROMANCE}             & \texttt{ROMANCEX}         \\
% \textbf{Character}  & Friendship                   & \texttt{FRIENDS}             & \texttt{FRIENDSX}         \\
% \textbf{Character}  & Former Romantic Relationship & \texttt{EXES}                & \texttt{EXESC}            \\
% \cline
% \multicolumn{2}{|l|}\textbf{Setting}               & \texttt{LOC}                 & \texttt{LOCX}             \\
% \multicolumn{2}{|l|}\textbf{Plot Point}            & \texttt{PLOT}                & \texttt{PLOTX}            \\
% \multicolumn{2}{|l|}\textbf{Research Item}         & \texttt{RSRCH}               & \texttt{RSRCHX}           \\
% \multicolumn{2}{|l|}\textbf{Revision Item}         & \texttt{REV}                 & \texttt{REVX}             \\
% \hline
% \end{tabular}
% \end{center}
%
%
% Each index creation macro takes a specific number of arguments as described
% in the following sections. The first argument of each macro is always a
% ``prefix'' which is added to the end of the prefix listed in the table above
% to form the name of the resulting index marking macro. For example, after
% including the commands:
%
% \begin{verbatim}
%   \IXCharacter{jane}{Jane}{Smith}
%   \IXCharacter{doug}{Doug}{Jones}
%   \IXSettingCityPlace{janeshouse}{Phoenix}{Jane's House}
%   \IXFriends{janedoug}{Jane}{Smith}{Doug}{Jones}
% \end{verbatim}
%
% The following index marking macros will exist for indexing:
%
% \begin{itemize}
%     \item The index marking macros \verb|\CHARjane| and \verb|\CHARXjane| for
%     adding index entries about Jane;
%     \item The index marking macros \verb|\CHARdoug| and \verb|\CHARXdoug| for
%     adding index entries about Doug;
%     \item The index marking macros \verb|\LOCjaneshouse| and
%     \verb|\LOCxjaneshouse| for adding index entries about Jane's house;
%     \item The index marking macros \verb|\FRIENDSjanedoug| and
%     \verb|\FRIENDSXjanedoug| for marking information about Jane \& Doug's
%     friendship.
% \end{itemize}
%
% The index creation macros are described in the following sections.
%
% \subsection{Indexing Characters}
%
% \DescribeMacro{\IXcharacter}
%
% The \verb|\IXCharacter| macro creates index marking commands for marking
% information about a specific character. It is the simplest form of index
% creation command for
% characters.
%
% \begin{description}
%     \item[Arguments]
%         \begin{enumerate}
%             \item Index-marking command tag (must be unique)
%             \item Character first name
%             \item Character last name
%         \end{enumerate}
%     \item[Notes]
% \end{description}
%
% \DescribeMacro{\IXAssociatedCharacter}
%
% The \verb|\IXAssociatedCharacter| macro creates index marking commands for
% marking information about a character associated with a specific organization.
%
% \begin{description}
%     \item[Arguments]
%         \begin{enumerate}
%             \item Index-marking command tag (must be unique)
%             \item Character first name
%             \item Character last name
%             \item Character association/organization name
%         \end{enumerate}
%     \item[Notes]
% \end{description}
%
% \DescribeMacro{\IXRomance}
%
% The \verb|\IXRomance| macro creates index marking commands for marking the fact
% that two characters have a romantic relationship.
%
% \begin{description}
%     \item[Arguments]
%         \begin{enumerate}
%             \item Index-marking command tag (must be unique)
%             \item First character's first name
%             \item First character's last name
%             \item Second character's first name
%             \item Second character's last name
%         \end{enumerate}
%     \item[Notes] The relationships are created reciprocally --- that is, index
%     entries are added to reflect that the first character is in a romantic
%     relationship with the second character, and also that the second character
%     is in a relationship with the first. Reciprocal entries are also added to
%     the plot point and master indexes.
% \end{description}
%
% \DescribeMacro{\IXFriends}
%
% The \verb|\IXFriends| macro creates index marking commands for marking the fact
% that two characters have a friendship.
%
% \begin{description}
%     \item[Arguments]
%         \begin{enumerate}
%             \item Index-marking command tag (must be unique)
%             \item First character's first name
%             \item First character's last name
%             \item Second character's first name
%             \item Second character's last name
%         \end{enumerate}
%     \item[Notes] The relationships are created reciprocally --- that is, index
%     entries are added to reflect that the first character is friends with the
%     second character, and also that the second character is friends with the
%     first. Reciprocal entries are also added to the plot point and master
%     indexes.
% \end{description}
%
% \DescribeMacro{\IXExes}
%
% The \verb|\IXFriends| macro creates index marking commands for marking the fact
% that two characters had a romantic relationship in the past.
%
% \begin{description}
%     \item[Arguments]
%         \begin{enumerate}
%             \item Index-marking command tag (must be unique)
%             \item First character's first name
%             \item First character's last name
%             \item Second character's first name
%             \item Second character's last name
%         \end{enumerate}
%     \item[Notes] The relationships are created reciprocally --- that is, index
%     entries are added to reflect that the first character is in a romantic
%     relationship with the second character, and also that the second character
%     is in a relationship with the first. Reciprocal entries are also added to
%     the plot point and master indexes.
% \end{description}
%
% \subsection{Indexing Places}
%
% \DescribeMacro{\IXSettingCity}
%
% The \verb|\IXSettingCity| macro creates index marking commands for marking
% references to a specific city.
%
% \begin{description}
%     \item[Arguments]
%         \begin{enumerate}
%             \item Index-marking command tag (must be unique)
%             \item City name (or city/state, etc.~)
%         \end{enumerate}
%     \item[Notes]
% \end{description}
%
% \DescribeMacro{\IXSettingCityPlace}
%
% The \verb|\IXSettingCityPlace| macro creates index marking commands for marking
% references to a specific place within a city.
%
% \begin{description}
%     \item[Arguments]
%         \begin{enumerate}
%             \item Index-marking command tag (must be unique)
%             \item City name (or city/state, etc.~)
%             \item Place name
%         \end{enumerate}
%     \item[Notes]
% \end{description}
%
% \DescribeMacro{\IXSettingCategoryPlace}
%
% The \verb|\IXSettingCategoryPlace| macro creates index marking commands for
% marking references to a specific type of place within a city. For example, if
% you wanted to index crime scene locations within your book, you could use this
% macro to tag them.
%
% \begin{description}
%     \item[Arguments]
%         \begin{enumerate}
%             \item Index-marking command tag (must be unique)
%             \item Setting category
%             \item City name (or city/state, etc.~)
%             \item Place name
%         \end{enumerate}
%     \item[Notes]
% \end{description}
%
%
% \DescribeMacro{\IXSettingCharacterPlace}
%
% The \verb|\IXSettingCharacterPlace| macro creates index marking commands for
% marking references to a place associated with a specific character.
%
% \begin{description}
%     \item[Arguments]
%         \begin{enumerate}
%             \item Index-marking command tag (must be unique)
%             \item Character first name
%             \item Character last name
%             \item City name (or city/state, etc.~)
%             \item Place name
%         \end{enumerate}
%     \item[Notes]
% \end{description}
%
% \subsection{Indexing Plot Points}
%
% \DescribeMacro{\IXPlotPoint}
%
% The \verb|\IXPlotPoint| macro creates index marking commands for
% marking references to a specific plot thread within your story.
%
% \begin{description}
%     \item[Arguments]
%         \begin{enumerate}
%             \item Index-marking command tag (must be unique)
%             \item Plot point description
%         \end{enumerate}
%     \item[Notes]
% \end{description}
%
% \DescribeMacro{\IXPlotSubPoint}
%
% The \verb|\IXPlotSubPoint| macro creates index marking commands for
% marking references to a specific sub-plot level plot thread within
% your story.
%
% \begin{description}
%     \item[Arguments]
%         \begin{enumerate}
%             \item Index-marking command tag (must be unique)
%             \item Plot point description
%             \item Plot sub-point description
%         \end{enumerate}
%     \item[Notes]
% \end{description}
%
% \DescribeMacro{\IXPlotSubSubPoint}
%
% The \verb|\IXPlotSubSubPoint| macro creates index marking commands for
% marking references to a specific sub-sub-plot level plot thread within
% your story.
%
% \begin{description}
%     \item[Arguments]
%         \begin{enumerate}
%             \item Index-marking command tag (must be unique)
%             \item Plot point description
%             \item Plot sub-point description
%             \item Plot sub-sub-point description
%         \end{enumerate}
%     \item[Notes]
% \end{description}
%
% \DescribeMacro{\IXCharacterPlotPoint}
%
% The \verb|\IXCharacterPlotPoint| macro creates index marking commands for
% marking references to a specific plot thread within your story, and also
% associates that plot thread with a specific character.
%
% \begin{description}
%     \item[Arguments]
%         \begin{enumerate}
%             \item Index-marking command tag (must be unique)
%             \item Character first name
%             \item Character last name
%             \item Plot point description
%         \end{enumerate}
%     \item[Notes]
% \end{description}
%
% \DescribeMacro{\IXLocatedPlotPoint}
%
% The \verb|\IXLocatedPlotPoint| macro creates index marking commands for
% marking references to a specific plot thread within your story, and also
% associates that plot thread with a specific location.
%
% \begin{description}
%     \item[Arguments]
%         \begin{enumerate}
%             \item Index-marking command tag (must be unique)
%             \item City name (or city/state, etc.~)
%             \item Place name
%             \item Plot point description
%         \end{enumerate}
%     \item[Notes]
% \end{description}
%
% \DescribeMacro{\IXLocatedSubPlotPoint}
%
% The \verb|\IXLocatedSubPlotPoint| macro creates index marking commands for
% marking references to a specific sub-plot level plot thread within your story,
% and also associates that plot thread with a specific location.
%
% \begin{description}
%     \item[Arguments]
%         \begin{enumerate}
%             \item Index-marking command tag (must be unique)
%             \item City name (or city/state, etc.~)
%             \item Place name
%             \item Plot point description
%             \item Plot sub-point description
%         \end{enumerate}
%     \item[Notes]
% \end{description}
%
%
% \DescribeMacro{\IXLocatedCharacterPlotPoint}
%
% The \verb|\IXLocatedCharacterPlotPoint| macro creates index marking commands for
% marking references to a specific plot thread within your story, and also
% associates that plot thread with a specific location and character.
%
% \begin{description}
%     \item[Arguments]
%         \begin{enumerate}
%             \item Index-marking command tag (must be unique)
%             \item City name (or city/state, etc.~)
%             \item Place name
%             \item Character first name
%             \item Character last name
%             \item Plot point description
%         \end{enumerate}
%     \item[Notes]
% \end{description}
%
% \DescribeMacro{\IXLocatedSubPlotPoint}
%
% The \verb|\IXLocatedSubPlotPoint| macro creates index marking commands for
% marking references to a specific sub-plot level plot thread within your story,
% and also associates that plot thread with a specific location.
%
% \begin{description}
%     \item[Arguments]
%         \begin{enumerate}
%             \item Index-marking command tag (must be unique)
%             \item City name (or city/state, etc.~)
%             \item Place name
%             \item Plot point description
%             \item Plot sub-point description
%         \end{enumerate}
%     \item[Notes]
% \end{description}
%
% \DescribeMacro{\IXLocatedCharacterSubPlotPoint}
%
% The \verb|\IXLocatedCharacterSubPlotPoint| macro creates index marking
% commands for marking references to a specific sub-plot level plot thread
% within your story, and also associates that plot thread with a specific
% location and character.
%
% \begin{description}
%     \item[Arguments]
%         \begin{enumerate}
%             \item Index-marking command tag (must be unique)
%             \item City name (or city/state, etc.~)
%             \item Place name
%             \item Character first name
%             \item Character last name
%             \item Plot point description
%             \item Plot sub-point description
%         \end{enumerate}
%     \item[Notes]
% \end{description}
%
% \subsection{Indexing Research and Revision Notes}
%
% \DescribeMacro{\IXResearch}
%
% The \verb|\IXResearch| macro creates index marking commands for
% marking places in your manuscript where additional research is needed.
%
% \begin{description}
%     \item[Arguments]
%         \begin{enumerate}
%             \item Index-marking command tag (must be unique)
%         \end{enumerate}
%     \item[Notes]
% \end{description}
%
%
% \DescribeMacro{\IXRevision}
%
% The \verb|\IXRevision| macro creates index marking commands for
% marking places in your story where revision is required.
%
% \begin{description}
%     \item[Arguments]
%         \begin{enumerate}
%             \item Index-marking command tag (must be unique)
%         \end{enumerate}
%     \item[Notes]
% \end{description}
%
% \section{Generating the Indexes}
%
% After you have marked all of the relevant indexing entries in your
% manuscript, all that remains is to output the indexes (or not). This is handled
% by the \verb|\PrintNovelIndexes| macro. I wrap the call to
% \verb|PrintNovelIndexes| inside a conditional so I can include the indexes in
% my personal author copy of my books but not in the print-ready PDFs I generate
% for publication.
%
% \DescribeMacro{\PrintNovelIndexes}
%
% The \verb|\PrintNovelIndexes| macro performs a \verb|\cleardoublepage| and then
% outputs the indexes. The character, plot, setting, revision, research, and
% master cross-reference indexes are all output, in that order. The macro performs
% a \verb|\cleardoublepage| before outputting the indexes.
%
% \section{Opportunities for Enhancement}
%
% The initial version of the package is very rough, and was built with just
% enough functionality for my needs. (I am nonetheless sharing it in the hope
% that it might be useful for others). Some obvious opportunities for
% improvement exist, including but not limited to:
%
% \begin{itemize}
%     \item Refactoring to reduce the amount of repetitive code;
%     \item Modifying \verb|\PrintNovelIndexes| to not output indexes which
%     don't contain any entries, and to give the user better control over the
%     formatting of the indexes;
%     \item Adding controls for the order in which the indexes are output;
%     \item Better control over how index entries are generated, including
%     allowing renaming/adjusting the indexing hierarchy;
%     \item More robust error handling. (Well, okay, \emph{any} error handling
%     would be an improvement at this point);
%     \item The ability to generate a reference list of all the index marking
%     macros defined by a given document.
% \end{itemize}
%
% The author welcomes contributions or enhancement suggestions via
% \url{https://github.com/tammymakesthings/novelidx.sty}{the GitHub repo} for
% the project.
%
% \section{Author's Note}
%
% I was first exposed to \LaTeX in college, more years ago than I care to admit.
% Although I can thus honestly say ``I have been a \LaTeX user for several
% decades now'', I have historically been ``guilty'' of learning the absolute
% minimum amount I needed to get by. It's only been very recently that I've made
% a concerted effort to raise my base level of knowledge.
%
% I am therefore confident there are many areas for improvement in this package.
% I'm publishing it anyway in the hope that it will be useful for others, and
% also in the hope that others like me will be encouraged to add their
% contributionsto the \TeX and \LaTeX community.

%
% \StopEventually{}
%
% \section{Implementation}
%
% \iffalse
%<*package>
% \fi
%
%%%%%%%%%%%%%%%%%%%%%%%%%%%%%%%%%%%%%%%%%%%%%%%%%%%%%%%%%%%%%%%%%%%%%%%%%%%%%%
%%%% novelidx: Helper commands for indexing novel metadata
%%%% Version 1.00, 2021-10-17, Tammy Cravit
%%%%%%%%%%%%%%%%%%%%%%%%%%%%%%%%%%%%%%%%%%%%%%%%%%%%%%%%%%%%%%%%%%%%%%%%%%%%%%
%
%    \begin{macrocode}
\NeedsTeXFormat{LaTeX2e}
\ProvidesPackage{novelidx}[2021-10-17 novel metadata indexing macros]
\RequirePackage{etoolbox}
\RequirePackage{imakeidx}
%    \end{macrocode}
%
% \begin{macro}{\SetupNovelIndexes}
%    \begin{macrocode}
\newcommand{\SetupNovelIndexes}{%
    \indexsetup{level=\chapter*,toclevel=chapter} %
    \makeindex[name=master,title=Master Cross Reference] %
    \makeindex[name=char,title=Character Index] %
    \makeindex[name=setting,title=Setting Index] %
    \makeindex[name=plot,title=Plot Point Index] %
    \makeindex[name=revision,title=Revision Notes] %
    \makeindex[name=research,title=Research Notes] %
}
%    \end{macrocode}
% \end{macro}
%
%
% \begin{macro}{\IXcharacter}
% tag, first, last
%    \begin{macrocode}
\newcommand{\IXcharacter}[3]{%
    \expandafter\def\csname CHAR#1\endcsname {%
        \protect\index[char]{#3, #2} %
        \protect\index[master]{Characters!#3, #2} %
    } %
    \expandafter\def\csname CHARX#1\endcsname ##1 {%
        \protect\index[char]{#3, #2!##1} %
        \protect\index[master]{Characters!#3, #2!##1} %
    }%
}
%    \end{macrocode}
% \end{macro}
%
%
% \begin{macro}{\IXAssociatedCharacter}
% tag, first, last, association
%    \begin{macrocode}
\newcommand{\IXAssociatedCharacter}[4]{%
    \expandafter\def\csname CHAR#1\endcsname {%
        \protect\index[char]{#4!#3, #2} %
        \protect\index[master]{Characters!#3, #2} %
    }
    \expandafter\def\csname CHARX#1\endcsname ##1 {%
        \protect\index[char]{#4!#3, #2!##1} %
        \protect\index[master]{Characters!#3, #2!##1} %
    }
}
%    \end{macrocode}
% \end{macro}
%
%
% \begin{macro}{\IXSettingCity}
% tag, city
%    \begin{macrocode}
\newcommand{\IXSettingCity}[2]{%
    \expandafter\def\csname LOC#1\endcsname {%
        \protect\index[setting]{#2} %
        \protect\index[master]{Cities!#2} %
    } %
    \expandafter\def\csname LOCX#1\endcsname ##1 {%
        \protect\index[setting]{#2!##1} %
        \protect\index[master]{Cities!#2!##1} %
    } %
}
%    \end{macrocode}
% \end{macro}
%
%
% \begin{macro}{\IXSettingCityPlace}
% tag, city, place
%    \begin{macrocode}
\newcommand{\IXSettingCityPlace}[3]{%
    \expandafter\def\csname LOC#1\endcsname { %
        \protect\index[setting]{#2!#3} %
        \protect\index[master]{Cities!#2} %
        \protect\index[master]{Locations!#2!#3} %
    } %
    \expandafter\def\csname LOCX#1\endcsname ##1 {%
        \protect\index[setting]{#2!#3!##1} %
        \protect\index[master]{Cities!#2!##1} %
        \protect\index[master]{Locations!#2!#3!##1} %
    } %
}
%    \end{macrocode}
% \end{macro}
%
%
% \begin{macro}{\IXSettingCategoryPlace}
% tag, category, city, place
%    \begin{macrocode}
\newcommand{\IXSettingCategoryPlace}[4]{ %
    \expandafter\def\csname LOC#1\endcsname { %
        \protect\index[setting]{#2!#3!#4} %
        \protect\index[setting]{#3!#2!#4} %
        \protect\index[master]{Cities!#3} %
        \protect\index[master]{Locations!#3!#4} %
        \protect\index[master]{Locations!#2!#3!#4} %
    } %
    \expandafter\def\csname LOCX#1\endcsname ##1 { %
        \protect\index[setting]{#2!#3!#4!##1} %
        \protect\index[setting]{#3!#2!#4!##1} %
        \protect\index[master]{Cities!#3!##1} %
        \protect\index[master]{Locations!#3!#4!##1}  %
        \protect\index[master]{Locations!#2!#3!#4 --- ##1}  %
    }
}
%    \end{macrocode}
% \end{macro}
%
%
% \begin{macro}{\IXSettingCharacterPlace}
% tag, first name, last name, city, place
%    \begin{macrocode}
\newcommand{\IXSettingCharacterPlace}[5] { %
    \expandafter\def\csname LOC#1\endcsname { %
        \protect\index[character]{#3, #2!#4!#5} %
        \protect\index[setting]{#4!#3, #2 - #5} %
        \protect\index[master]{Cities!#4} %
        \protect\index[master]{Locations!#4!#3, #2 - #5} %
        \protect\index[master]{Character Locations!#3, #2!#4!#5} %
    } %
    \expandafter\def\csname LOCX#1\endcsname ##1 { %
        \protect\index[character]{#3, #2!#4!#5!##1} %
        \protect\index[setting]{#4!#3, #2 - #5!##1} %
        \protect\index[master]{Cities!#4!##1} %
        \protect\index[master]{Locations!#4!#3, #2 - #5!##1} %
        \protect\index[master]{Character Locations!#3, #2!#4!#5 - ##1} %
    } %
}
%    \end{macrocode}
% \end{macro}
%
%
% \begin{macro}{\IXPlotPoint}
% tag, plot point
%    \begin{macrocode}
\newcommand{\IXPlotPoint}[2]{%
    \expandafter\def\csname PLOT#1\endcsname { %
        \protect\index[plot]{#2} %
        \protect\index[master]{Plot Points!#2}  %
    } % 
    \expandafter\def\csname PLOTX#1\endcsname ##1 {%
        \protect\index[plot]{#2} %
        \protect\index[master]{Plot Points!#2!##1} %
    } %
}
%    \end{macrocode}
% \end{macro}
%
%
% \begin{macro}{\IXPlotSubPoint}
% tag, plot point, plot sub point
%    \begin{macrocode}
\newcommand{\IXPlotSubPoint}[3]{ %
    \expandafter\def\csname PLOT#1\endcsname { %
        \protect\index[plot]{#2!#3} %
        \protect\index[master]{Plot Points!#2!#3} %
    } %
    \expandafter\def\csname PLOTX#1\endcsname ##1 { %
        \protect\index[plot]{#2!#3!##1} %
        \protect\index[master]{Plot Points!#2!#3!##1} %
    } %
}
%    \end{macrocode}
% \end{macro}
%
%
% \begin{macro}{\IXPlotSubSubPoint}
% tag, plot point, plot sub point, plot sub sub point
%    \begin{macrocode}
\newcommand{\IXPlotSubSubPoint}[4]{ %
    \expandafter\def\csname PLOT#1\endcsname { %
        \protect\index[plot]{#2!#3!#4} %
        \protect\index[master]{Plot Points!#2!#3!#4} %
    } %
    \expandafter\def\csname PLOTX#1\endcsname ##1 { %
        \protect\index[plot]{#2!#3!#4!##1} %
        \protect\index[master]{Plot Points!#2!#3!#4 - ##1} %
    } %
} 
%    \end{macrocode}
% \end{macro}
%
%
% \begin{macro}{\IXCharacterPlotPoint}
% tag, first name, last name, plot point
%    \begin{macrocode}
\newcommand{\IXCharacterPlotPoint}[4]{ %
    \expandafter\def\csname PLOT#1\endcsname { %
        \protect\index[char]{#3, #2!#4} %
        \protect\index[plot]{#4} %
        \protect\index[master]{Characters!#3, #2!#4} %
        \protect\index[master]{Plot Points!#4} %
    } %
    \expandafter\def\csname PLOTX#1\endcsname ##1 { %
        \protect\index[char]{#3, #2!#4!Plot - ##1} %
        \protect\index[plot]{#4!##1} %
        \protect\index[master]{Plot Points!#4!##1} %
        \protect\index[master]{Characters!#3, #2!#4} %
    }
}
%    \end{macrocode}
% \end{macro}
%
%
% \begin{macro}{\IXRomance}
% tag, first name A, last name A, first name B, last name B
%    \begin{macrocode}
\newcommand{\IXRomance}[5]{ %
    \expandafter\def\csname ROMANCE#1\endcsname { %
        \protect\index[char]{#3, #2!romantic relationships!with #5, #4} %
        \protect\index[plot]{Romantic Relationships!#3, #2!with #5, #4} %
        \protect\index[master]{Characters!#3, #2!romantic relationships!with #5, #4} %
        \protect\index[master]{Plot Points!Romantic Relationships!#3, #2!with #5, #4} %
        \protect\index[master]{Relationships!#3, #2!romance with #5, #4} %
        \protect\index[char]{#5, #4!romantic relationships!with #3, #2} %
        \protect\index[plot]{Romantic Relationships!#5, #4!with #3, #2} %
        \protect\index[master]{Characters!#5, #4!romantic relationships!with #3, #2} %
        \protect\index[master]{Plot Points!Romantic Relationships!#5, #4!with #5, #2} %
        \protect\index[master]{Relationships!#5, #4!romance with #3, #2} %
    }  %
    \expandafter\def\csname ROMANCEX#1\endcsname ##1 { %
        \protect\index[char]{#3, #2!romantic relationships!with #5, #4!##1} %
        \protect\index[plot]{Romantic Relationships!#3, #2!with #5, #4!##1} %
        \protect\index[master]{Characters!#3, #2!romantic relationships!with #5, #4 - ##1} %
        \protect\index[master]{Plot Points!Romantic Relationships!#3, #2!with #5, #4 - ##1} %
        \protect\index[master]{Relationships!#3, #2!romance with #5, #4!##1} %
        \protect\index[char]{#5, #4!romantic relationships!with #3, #2!##1} %
        \protect\index[plot]{Romantic Relationships!#5, #4!with #3, #2!##1} %
        \protect\index[master]{Characters!#5, #4!romantic relationships!with #3, #2 - ##1} %
        \protect\index[master]{Plot Points!Romantic Relationships!#5, #4!with #3, #2 - ##1} %
        \protect\index[master]{Relationships!#5, #4!romance with #3, #2!##1} %
    } %
}
%    \end{macrocode}
% \end{macro}
%
%
% \begin{macro}{\IXFriends}
%    \begin{macrocode}
\newcommand{\IXFriends}[5]{ %
    \expandafter\def\csname FRIENDS#1\endcsname { %
        \protect\index[char]{#3, #2!friendships!with #5, #4} %
        \protect\index[plot]{Friendships!#3, #2!with #5, #4} %
        \protect\index[master]{Characters!#3, #2!friendships!with #5, #4} %
        \protect\index[master]{Plot Points!Friendships!#3, #2!with #5, #4} %
        \protect\index[master]{Relationships!#3, #2!friendship with #5, #4} %
        \protect\index[char]{#5, #4!friendships!with #3, #2} %
        \protect\index[plot]{Friendships!#5, #4!with #3, #2} %
        \protect\index[master]{Characters!#5, #4!friendships!with #3, #2} %
        \protect\index[master]{Plot Points!Friendships!#5, #4!with #5, #2}  %
        \protect\index[master]{Relationships!#5, #4!friendship with #3, #2} %
    }  %
    \expandafter\def\csname FRIENDSX#1\endcsname ##1 { %
        \protect\index[char]{#3, #2!friendships!with #5, #4!##1} %
        \protect\index[plot]{Friendships!#3, #2!with #5, #4!##1} %
        \protect\index[master]{Characters!#3, #2!friendships!with #5, #4 - ##1} %
        \protect\index[master]{Plot Points!Friendships!#3, #2!with #5, #4 - ##1} %
        \protect\index[master]{Relationships!#3, #2!friendship with #5, #4!##1} %
        \protect\index[char]{#5, #4!friendships!with #3, #2!##1} %
        \protect\index[plot]{Friendships!#5, #4!with #3, #2!##1} %
        \protect\index[master]{Characters!#5, #4!friendships!with #3, #2 - ##1} %
        \protect\index[master]{Plot Points!Friendships!#5, #4!with #3, #2 - ##1} %
        \protect\index[master]{Relationships!#5, #4!friendship with #3, #2!##1} %
    } %
}
%    \end{macrocode}
% \end{macro}
%
%
% \begin{macro}{\IXExes}
%    \begin{macrocode}
\newcommand{\IXExes}[5]{ %
    \expandafter\def\csname EXES#1\endcsname { %
        \protect\index[char]{#3, #2!former relationships!with #5, #4} %
        \protect\index[plot]{Former Relationships!#3, #2!with #5, #4} %
        \protect\index[master]{Characters!#3, #2!former relationships!with #5, #4} %
        \protect\index[master]{Plot Points!Former Relationships!#3, #2!with #5, #4} %
        \protect\index[master]{Relationships!#3, #2!former relationship with #5, #4} %
        \protect\index[char]{#5, #4!former relationships!with #3, #2} %
        \protect\index[plot]{Former Relationship!#5, #4!with #3, #2}% %
        \protect\index[master]{Characters!#5, #4!former relationships!with #3, #2} %
        \protect\index[master]{Plot Points!Former Relationship!#5, #4!with #5, #2} %
        \protect\index[master]{Relationships!#5, #4!former relationship with #3, #2} %
    }  %
    \expandafter\def\csname EXESX#1\endcsname ##1 { %
        \protect\index[char]{#3, #2!former relationship!with #5, #4!##1} %
        \protect\index[plot]{Former Relationship!#3, #2!with #5, #4!##1} %
        \protect\index[master]{Characters!#3, #2!former relationship!with #5, #4 - ##1} %
        \protect\index[master]{Plot Points!Former Relationship!#3, #2!with #5, #4 - ##1} %
        \protect\index[master]{Relationships!#3, #2!former relationship with #5, #4!##1} %
        \protect\index[char]{#5, #4!former relationship!with #3, #2!##1} %
        \protect\index[plot]{Former Relationship!#5, #4!with #3, #2!##1} %
        \protect\index[master]{Characters!#5, #4!former relationship!with #3, #2 - ##1} %
        \protect\index[master]{Plot Points!Former Relationship!#5, #4!with #3, #2 - ##1} %
        \protect\index[master]{Relationships!#5, #4!former relationship with #3, #2!##1} %
    } %
}
%    \end{macrocode}
% \end{macro}
%
%
% \begin{macro}{\IXLocatedPlotPoint}
% tag, city, location, plot point
%    \begin{macrocode}
\newcommand{\IXLocatedPlotPoint}[4]{ %
    \expandafter\def\csname PLOT#1\endcsname { %
        \protect\index[char]{#3, #2!#4} %
        \protect\index[plot]{#4} %
        \protect\index[master]{Cities!#2} %
        \protect\index[master]{Locations!#2!#3} %
        \protect\index[master]{Plot Points!#4} %
        \protect\index[master]{Plot Locations!#2!#4} %
    } %
    \expandafter\def\csname PLOTX#1\endcsname ##1 { %
        \protect\index[char]{#3, #2!#4} %
        \protect\index[plot]{#4} %
        \protect\index[master]{Cities!#2}  %
        \protect\index[master]{Locations!#2!#3} %
        \protect\index[master]{Plot Points!#4}  %
        \protect\index[master]{Plot Locations!#2!#4} %
    } %
}
%    \end{macrocode}
% \end{macro}
%
%
% \begin{macro}{\IXLocatedSubPlotPoint}
% tag, city, location, plot point, plot sub point
%    \begin{macrocode}
\newcommand{\IXLocatedSubPlotPoint}[5]{ %
    \expandafter\def\csname PLOT#1\endcsname { %
        \protect\index[char]{#3, #2!#4!#5} %
        \protect\index[plot]{#4!#5} %
        \protect\index[master]{Cities!#2} %
        \protect\index[master]{Locations!#2!#3} %
        \protect\index[master]{Plot Points!#4!#5} %
        \protect\index[master]{Plot Locations!#2!#4!#5} %
    } %
    \expandafter\def\csname PLOTX#1\endcsname ##1 { %
        \protect\index[char]{#3, #2!#4!#5!##1} %
        \protect\index[plot]{#4!#5!##1} % 
        \protect\index[master]{Cities!#2!##1} %
        \protect\index[master]{Locations!#2!#3!##1} %
        \protect\index[master]{Plot Points!#4!#5!##1} %
        \protect\index[master]{Plot Locations!#2!#4!#5 - ##1} %
    } %
} 
%    \end{macrocode}
% \end{macro}
%
%
% \begin{macro}{\IXLocatedCharacterPlotPoint}
% tag, city, place name, first name, last name, plot point
%    \begin{macrocode}
\newcommand{\IXLocatedCharacterPlotPoint}[6]{ %
    \expandafter\def\csname PLOT#1\endcsname { %
        \protect\index[char]{#5, #4!#2!#6} %
        \protect\index[plot]{#5, #4!#6} % 
        \protect\index[master]{Cities!#2} %
        \protect\index[master]{Plot Points!#5, #4!#6} %
        \protect\index[master]{Plot Locations!#2!#3 (#5, #4)!#6} %
    }
    \expandafter\def\csname PLOTX#1\endcsname ##1 {    
        \protect\index[char]{#5, #4!#2!#6!##1} %
        \protect\index[plot]{#5, #4!#6!##1} % 
        \protect\index[master]{Cities!#2!##1} %
        \protect\index[master]{Plot Points!#5, #4!#6!##1} %
        \protect\index[master]{Plot Locations!#2!#3 (#5, #4)!#6 - ##1} %
    } %
}
%    \end{macrocode}
% \end{macro}
%
%
% \begin{macro}{\IXLocatedCharacterSubPlotPoint}
% tag, city, location, first name, last name, plot point, sub plot point
%    \begin{macrocode}
\newcommand{\IXLocatedCharacterSubPlotPoint}[7]{ %
    \expandafter\def\csname PLOT#1\endcsname { %
        \protect\index[char]{#5, #4!#2!#6!#7} %
        \protect\index[plot]{#5, #4!#6!#7} % 
        \protect\index[master]{Cities!#2} %
        \protect\index[master]{Plot Points!#5, #4!#6!#7} %
        \protect\index[master]{Plot Locations!#2!#3 (#5, #4)!#6!#7} %
    }
    \expandafter\def\csname PLOTX#1\endcsname ##1 {  %
        \protect\index[char]{#5, #4!#2!#6!##1} %
        \protect\index[plot]{#5, #4!#6!##1} % 
        \protect\index[master]{Cities!#2!##1} %
        \protect\index[master]{Plot Points!#5, #4!#6!##1} %
        \protect\index[master]{Plot Locations!#2!#3 (#5, #4)!#6 - #7 - ##1} %
    } %
}
%    \end{macrocode}
% \end{macro}
%
%
% \begin{macro}{\IXResearch}
%    \begin{macrocode}
\newcommand{\IXResearch}[2]{ %
    \expandafter\def\csname RSRCH#1\endcsname { %
        \protect\index[research]{#1} %
        \protect\index[master]{Research Notes!#1} %
    } %
    \expandafter\def\csname RSRCHX#1\endcsname ##1 { %
        \protect\index[research]{#1!##1} %
        \protect\index[master]{Research Notes!#1!##1} %
    }     %
}
%    \end{macrocode}
% \end{macro}
%
%
% \begin{macro}{\IXRevision}
%    \begin{macrocode}
\newcommand{\IXRevision}[2]{  %
    \expandafter\def\csname REV#1\endcsname { %
        \protect\index[revision]{#1} %
        \protect\index[master]{Revision Notes!#1} %
    } %
    \expandafter\def\csname REVX#1\endcsname ##1 { %
        \protect\index[revision]{#1!##1} %
        \protect\index[master]{Revision Notes!#1!##1} %
    } % 
}
%    \end{macrocode}
% \end{macro}
%
%
% \begin{macro}{\PrintNovelIndexes}
%    \begin{macrocode}
\newcommand{\PrintNovelIndexes}[0]{ %
    \cleardoublepage %
    \printindex[char] %
    \printindex[setting] %
    \printindex[plot] %
    \printindex[revision] %
    \printindex[research] %
    \printindex[master] %
}
%    \end{macrocode}
% \end{macro}
%
%    \begin{macrocode}
\AtEndPreamble{ %
    \SetupNovelIndexes %
}
%    \end{macrocode}

%
% \iffalse
%</package>
% \fi
%
% \Finale
\endinput
